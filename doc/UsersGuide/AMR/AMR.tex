Our approach to adaptive refinement in CASTRO uses a nested
hierarchy of logically-rectangular grids with simultaneous refinement
of the grids in both space and time.  The integration algorithm on the grid hierarchy
is a recursive procedure in which coarse grids are advanced in time,
fine grids are advanced multiple steps to reach the same time
as the coarse grids and the data at different levels are then synchronized.

During the regridding step, increasingly finer grids
are recursively embedded in coarse grids until the solution is
sufficiently resolved.  An error estimation procedure based on
user-specified criteria evaluates where additional refinement is needed
and grid generation procedures dynamically create or
remove rectangular fine grid patches as resolution requirements change.

A good introduction to the style of AMR used here is in Lecture 1
of the Adaptive Mesh Refinement Short Course at
https://ccse.lbl.gov/people/jbb/index.html

\section{Synchronization Algorithm}

Here is the AMR algorithm for the compressible equations with
self-gravity.  The gravity component of the algorithm 
is closely related to (but not identical to) that in Miniati and Colella,
JCP, 2007 (in press).

Over a coarse grid time step 
we collect flux register information for the hyperbolic part of the synchronization:
\begin{equation}
\delta\Fb = -\Delta t_c A^c F^c + \sum \Delta t_f A^f F^f
\end{equation}
Analogously, at the end of a coarse grid time step 
we store the mismatch in normal gradients of $\phi$ at the coarse-fine interface:
\begin{equation}
\delta F_\phi =  - A^c \frac{\partial \phi^c}{\partial n}
+ \sum A^f \frac{\partial \phi^f}{\partial n}
\end{equation}
Because we want the composite $\phi^{c-f}$ to satisfy the multilevel version of  (\ref{eq:Self Gravity}) at each time $t^n$, we do not accumulate $\frac{\partial \phi^f}{\partial n}$ over time, rather we add coarse and fine fluxes only at integer coarse times.

At the end of a coarse grid time step we can define
${\overline{\Ub}}^{c-f}$ and $\overline{\phi}^{c-f}$
as the composite of the data from coarse and fine grids as a provisional solution at
time $n+1$. (Assume $\overline{\Ub}$ has been averaged down so that the data on coarse cells
underlying fine cells is the average of the fine cell data above it.)

The synchronization consists of two parts: 

\begin{itemize}
\item Step 1:  Hyperbolic reflux

In the hyperbolic reflux step, we update the conserved variables with the flux synchronization
and adjust the gravitational terms to reflect the changes in $\rho$ and $\ub$.
\begin{equation}
{\Ub}^{c, \star} = \overline{\Ub}^{c} + \frac{\delta\Fb}{V},
\end{equation}
where $V$ is the volume of the cell and the correction from $\delta\Fb$ is supported only on coarse cells adjacent to fine grids.   

\item Step 2:  Gravitational synchronization

In this step we correct for the mismatch in normal derivative in $\phi^{c-f}$ at the coarse-fine
interface, as well as accounting for the changes in source terms for $(\rho \ub)$
and $(\rho E)$ due to the change in $\rho.$

On the coarse grid only, we define
\begin{equation}
(\delta \rho)^{c} =  \rho^{c, \star} - {\overline{\rho}}^{c}  .
\end{equation}

We then form the composite residual, which is composed of two contributions.
The first is the degree to which the current $ \overline{\phi}^{c-f}$
does not satisfy the original equation on a composite grid (since we have solved for 
$\overline{\phi}^{c-f}$ separately on the coarse and fine levels).  The second
is the response of $\phi$ to the change in $\rho.$
We define 
\begin{equation} R \equiv  - 4 \pi G \rho^{\star,c-f} - \Delta^{c-f} \; \overline{\phi}^{c-f} 
= - 4 \pi G (\delta \rho)^c - (\nabla \cdot \delta F_\phi ) |_c   .
\end{equation}

Then we solve
\begin{equation}
 \Delta^{c-f} \; \delta \phi^{c-f} = R
\label{eq:gravsync}
\end{equation}
as a two level solve at the coarse and fine levels.  
We define the update to gravity,
\begin{equation}
\delta {\bf g}^{c-f} = \nabla (\delta \phi^{c-f})  .
\end{equation}

Define the \sync sources for momentum on the coarse and fine, $S^{\sync,c}_{\rho \ub}$, and
$S^{\sync,f}_{\rho \ub}$, respectively as follows:
\begin{eqnarray}
S^{\sync,c}_{\rho \ub} &=& \left(
        \overline{\rho}^c + (\delta \rho)^c \right) ({\bf g}^{c,n+1} + \delta {\bf g}^{c-f}) -
        \overline{\rho}^c                 \; {\bf g}^{c,n+1} \nonumber \\
           &=& \left[( \delta \rho)^c {\bf g}^{c,n+1} + 
                \rho^{c,\star} \delta {\bf g}^c)  \right ] \\
S^{\sync,f}_{\rho \ub} &=& \overline{\rho}^{f} \; \delta {\bf g}^f  .
\end{eqnarray}
These momentum sources lead to the following energy sources:
\begin{eqnarray}
S^{\sync,c}_{\rho E} &=& S^{\sync,c}_{\rho \ub} \cdot
         \left(   \overline{\ub}^{c,n+1} + S^{\sync,c}_{\rho \ub} \Delta t_c / (2 \overline{\rho}^{c,n+1}) \right) \\
S^{\sync,f}_{\rho E} &=& S^{\sync,f}_{\rho \ub} \cdot
         \left(
    \overline{\ub}^{f} + S^{\sync,f}_{\rho \ub} \Delta t_f / (2 \; \overline{\rho}^{f} )
         \right)
\end{eqnarray}

The coarse and fine level state is updated using:
\begin{eqnarray}
(\rho \ub)^{c,n+1} =  (\rho \ub)^{c, \star } + \myhalf \Delta t_c S^{\sync,c}_{\rho \ub} ,&&
(\rho \ub)^{f,n+1} =  (\overline{\rho \ub})^{f} + \myhalf \Delta t_f S^{\sync,f}_{\rho \ub},  \\
(\rho E)^{c,n+1} =  (\rho E)^{c, \star } + \myhalf \Delta t_c S^{\sync,c}_{\rho E} ,&&
(\rho E)^{f,n+1} =  (\overline{\rho E})^{f} + \myhalf \Delta t_f S^{\sync,f}_{\rho E}  .
\end{eqnarray}
As the final component of this step we need to 
\begin{itemize}
\item add $\delta \phi^{c-f}$ directly to 
to $\phi^{c}$ and $\phi^{f}$ and interpolate $\delta \phi^{c-f}$ to any finer
levels and add to the current $\phi$ at those levels.

\item if level $c$ is not the coarsest level in the calculation, then we must transmit the
effect of this change in $\phi$ to the coarser levels by updating the flux register between
level $c$ and the next coarser level, $cc.$ In particular, we set
\begin{equation}
\delta {F_\phi}^{cc-c} = \delta F_\phi^{cc-c} 
+ \sum A^c \frac{\partial (\delta \phi)^{c-f}}{\partial n}  .
\end{equation}
\end{itemize}

\end{itemize}
