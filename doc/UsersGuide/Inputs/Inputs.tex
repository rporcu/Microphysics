The \mfix\ executable reads run-time information from an "inputs" file (which you put on the command line) 
as well as from {\tt mfix.dat}.  The following parameters are set in the inputs file

\section{Problem Geometry}

\subsection{List of Parameters}

\begin{table*}[h]
\begin{scriptsize}
\begin{tabular}{|l|l|l|l|} \hline
Parameter & Definition & Acceptable Values & Default \\
\hline
{\bf geometry.prob\_lo} & physical location of low corner of the domain & Real & must be set\\
{\bf geometry.prob\_hi} & physical location of high corner of the domain & Real  & must be set\\
{\bf geometry.coord\_sys} & coordinate system & 0 = Cartesian, 1 = r-z, 2 = spherical & must be set \\
{\bf geometry.is\_periodic} & is the domain periodic in this direction & 0 if false, 1 if true  & 0 0 0 \\
\hline
\end{tabular}
\label{Table:Geometry}
\end{scriptsize}
\end{table*}

\subsection{Examples of Usage}

\begin{itemize}

\item {\bf geometry.prob\_lo} = 0 0 0 \\
      {\bf geometry.prob\_hi} = 1.e8 2.e8 2.e8 \\
defines the  low corner of the domain to be (0.,0.,0.)       in physical space and the
        the high corner of the domain to be (1.e8,2.e8,2.e8) in physical space.  

\item {\bf geometry.coord\_sys} = 0 \\
defines the coordinate system as Cartesian 

\item {\bf geometry.is\_periodic} = 0 1 0 \\
says the domain is periodic in the y-direction only. 

\end{itemize}

\section{Resolution and Gridding}
\subsection{List of Parameters}

\begin{table*}[h]
\begin{scriptsize}
\begin{center}
\begin{tabular}{|l|l|l|l|} \hline
Parameter & Definition & Acceptable Values &Default\\
\hline
{\bf amr.max\_level}         &  finest possible level (only relevant if we were doing AMR)     & Integer        & must be set \\
{\bf amr.n\_cell}            &  number of cells in each coordinate direction of the domain     & Integers $> 0$ & must be set \\
{\bf amr.max\_grid\_size}    &  maximum allowable number of cells of one grid in any direction & Integer        & 32 \\
{\bf amr.max\_grid\_size\_x} &  maximum allowable number of cells of one grid in x-direction   & Integer        & 32 \\
{\bf amr.max\_grid\_size\_y} &  maximum allowable number of cells of one grid in y-direction   & Integer        & 32 \\
{\bf amr.max\_grid\_size\_z} &  maximum allowable number of cells of one grid in z-direction   & Integer        & 32 \\ \\
{\bf fabarray.mfiter\_tile\_size} & tile size for mesh data     & Integers       & (1024000,8,8) \\
{\bf particles.tile\_size}        & tile size for particle data & Integer        & (1024000,8,8) \\
\hline
\end{tabular}
%\caption{Input Parameters -- Resolution}
\label{Table:ResInputs}
\end{center}
\end{scriptsize}
\end{table*}

\subsection{Notes}

\begin{itemize}

\item Either one value of {\bf amr.max\_grid\_size} should be specified that will be used for
all coordinate directions,  or each of 
{\bf amr.max\_grid\_size\_x}, {\bf amr.max\_grid\_size\_y} and {\bf amr.max\_grid\_size\_z}
should be specified independently.

\item If {\bf fabarray.mfiter\_tile\_size} is included in the inputs file, three integers (one for each coordinate direction)
must be specified.  

\item Similarly, if {\bf particles.tile\_size} is included in the inputs file, three integers (one for each coordinate direction)
must be specified.  

\end{itemize}

\subsection{Examples of Usage}

\begin{itemize}

\item {\bf amr.n\_cell} = 32 64 64

would define the domain to have 32 cells in the x-direction, 64 cells in the y-direction, 
and 64 cells in the z-direction.

\item {\bf amr.max\_grid\_size} = 16

would break a $32^3$ domain into eight $16^3$ boxes.

\item {\bf amr.max\_grid\_size\_x} = 16 \\
      {\bf amr.max\_grid\_size\_y} =  8 \\
      {\bf amr.max\_grid\_size\_z} = 32

would break a $32^3$ domain into 8 $16$x$8$x$32$ boxes.

\end{itemize}

\section{Restart Capability}

\mfix\ has checkpointing and restarting capability using the highly efficient \amrex\ parallel I/O.
The following options in the inputs file control the generation of checkpoint files (which are really
directories):\\

\subsection{List of Parameters}

\begin{table*}[h]
\begin{scriptsize}
\begin{center}
\begin{tabular}{|l|l|l|l|} \hline
Parameter & Definition & Acceptable Values &Default\\
\hline
{\bf amr.check\_file} & prefix for restart files & Text & "chk" \\
{\bf amr.check\_int}  & how often to write restart files & Integer $> 0$ & -1  \\
{\bf amr.restart}     & name of the file (directory) from which to restart & Text & not used if not set \\
\hline
\end{tabular}
\end{center}
\end{scriptsize}
\end{table*}

\subsection{Examples of Usage}

\begin{itemize}

\item {\bf amr.check\_file} = chk\_run
\item {\bf amr.check\_int} = 10

means that restart files (really directories) starting with the prefix "chk\_run" will be
generated every 10 time steps.  The directory names will be {\it chk\_run00000}, 
{\it chk\_run00010}, {\it chk\_run00020}, etc.

\end{itemize}

To restart from {\it chk\_run00020}, for example, then set 

\begin{itemize}
\item {\bf amr.restart} = chk\_run00020
\end{itemize}

\section{Controlling PlotFile Generation}
\label{sec:PlotFiles}
The main output from \mfix\ is in the form of plotfiles (which are really directories).
The following options in the inputs file control the generation of plotfiles 

\subsection{List of Parameters}

\begin{table*}[h]
\begin{scriptsize}
\begin{center}
\begin{tabular}{|l|l|l|l|} \hline
Parameter & Definition & Acceptable Values &Default\\
\hline
{\bf amr.plot\_file} & prefix for plotfiles & Text & "plt" \\
{\bf amr.plot\_int}  & how often (by level 0 time steps) to write plot files & Integer $> 0$ & -1  \\
{\bf fab.format}     & Should we write the plotfile in double or single precision? & NATIVE or IEEE32 & NATIVE \\
\hline
\end{tabular}
\end{center}
\end{scriptsize}
\end{table*}

\subsection{Notes}

\begin{itemize}

\item By default, plotfiles are written in double precision (NATIVE format).  If you want to 
save space by writing them in single precision, set the fab.format flag to IEEE32.

\end{itemize}

\subsection{Examples of Usage}

\begin{itemize}

\item {\bf amr.plot\_file} = plt\_run
\item {\bf amr.plot\_int} = 10

means that plot files (really directories) starting with the prefix "plt\_run" will be
generated every 10 level 0 time steps.  The directory names will be {\it plt\_run00000}, {\it plt\_run00010}, {\it plt\_run00020}, etc.  These plotfiles can be visualized with the
tools described further in Chapter 9 of the \amrex\ User's Guide.

\end{itemize}

\section{Writing ASCII Particle Files}
\label{sec:AsciiFiles}
Sometimes it's simplest for debugging purposes to write out all the particles in a single ASCII file.
The following options in the inputs file control the writing of these particle files.

\subsection{List of Parameters}

\begin{table*}[h]
\begin{scriptsize}
\begin{center}
\begin{tabular}{|l|l|l|l|} \hline
Parameter & Definition & Acceptable Values &Default\\
\hline
{\bf amr.par\_ascii\_file} & prefix for ASCII particle files & Text & "par" \\
{\bf amr.par\_ascii\_int}  & how often to write ASCII particle files & Integer $> 0$ & -1  \\
\hline
\end{tabular}
\end{center}
\end{scriptsize}
\end{table*}

\subsection{Examples of Usage}

\begin{itemize}

\item {\bf amr.par\_ascii\_file} = DEM\_par
\item {\bf amr.par\_ascii\_int} = 10

means that ASCII particle files starting with the prefix "DEM\_par" will be
generated every 10 time steps.  The file names will be 
{\it DEM\_par00000}, {\it DEM\_par00010}, {\it DEM\_par00020}, etc.  

\end{itemize}


