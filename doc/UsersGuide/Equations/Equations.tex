\section{Conservation Forms}
We begin with the fully compressible equations for the conserved state vector, 
$\Ub = (\rho, \rho \ub, \rho E, \rho A_k, \rho X_k, \rho Y_k):$
\begin{eqnarray}
\frac{\partial \rho}{\partial t} &=& - \nabla \cdot (\rho \ub) + S_{{\rm ext},\rho}, \\
\frac{\partial (\rho \ub)}{\partial t} &=& - \nabla \cdot (\rho \ub \ub) - \nabla p +\rho \gb + \Sb_{{\rm ext},\rho\ub}, \\
\frac{\partial (\rho E)}{\partial t} &=& - \nabla \cdot (\rho \ub E + p \ub) + \rho \ub \cdot \gb - \sum_k {\rho q_k \dot\omega_k} + \nabla\cdot\kappa\nabla T + S_{{\rm ext},\rho E}, \\
\frac{\partial (\rho A_k)}{\partial t} &=& - \nabla \cdot (\rho \ub A_k) + S_{{\rm ext},\rho A_k}, \\
\frac{\partial (\rho X_k)}{\partial t} &=& - \nabla \cdot (\rho \ub X_k) + \rho \dot\omega_k + S_{{\rm ext},\rho X_k}, \\
\frac{\partial (\rho Y_k)}{\partial t} &=& - \nabla \cdot (\rho \ub Y_k) + S_{{\rm ext},\rho Y_k}.
\end{eqnarray}
Here $\rho, \ub, T, p$, and $\kappa$ are the density, velocity,
temperature, pressure, and thermal conductivity, respectively, and 
$E = e + \ub \cdot \ub / 2$ is the total energy with $e$ representing the internal
energy.  In addition, $X_k$ is the abundance of the $k^{\rm th}$ isotope,
with associated production rate, $\dot\omega_k$, and
energy release, $q_k$.  Here $\gb$ is the gravitational vector,
and $S_{{\rm ext},\rho}, \Sb_{{\rm ext}\rho\ub}$, etc.,
are user-specified source terms.  $A_k$ is an advected quantity, i.e., 
a tracer.  We also carry around auxiliary variables, $Y_k$, which have a 
user-defined evolution equation, but by default are treated as advected
quantities.

In the code we also carry around $T$ and $\rho e$ in the conservative state vector
even though they are derived from the other conserved quantities.
The ordering of the elements within $\Ub$ is defined (in 3D) by
\begin{itemize}
\item URHO: $\rho$
\item UMX: $\rho u$
\item UMY: $\rho v$
\item UMZ: $\rho w$
\item UEDEN: $\rho E$
\item UEINT: $\rho e$ - this is computed from the other quantities using
             $\rho e = \rho E - \rho \ub \cdot \ub / 2$.
\item UTEMP: $T$ - this is computed from the other quantities using the EOS
\item UFA: $\rho A_1$, the first advected quantity
\item UFS: $\rho X_1$, the first species
\item UFX: $\rho Y_1$, the first auxiliary variable
\end{itemize}

There are {\it nadv} advected quantities, which range from UFA: UFA+{\it nadv}-1.
In addition, there are {\it nspec} species (defined in the network directory),  which
range from UFS: UFS+{\it nspec}-1. Finally, there are {\it naux}
auxiliary variables, from UFX:UFX+{\it naux}-1, and {\it nadv} advected
quantities, which range from UFA: UFA + {\it nadv} - 1.  
The advected quantities have no effect at all on the rest of the solution
but can be useful as tracer quantities.  The auxiliary
variables are passed into the equation of state routines along with the
species;  An example of an auxiliary variable is the electron fraction, 
Ye, in core collapse simulations.

\section{Primitive Forms}
Here are the primitive forms of the equations for the primitive state vector,
$\Qb = (\rho, \ub, \rho e, A_k, X_k, Y_k)$:
\begin{eqnarray}
\frac{\partial\rho}{\partial t} &=& -\ub\cdot\nabla\rho - \rho\nabla\cdot\ub + S_{{\rm ext},\rho}, \\
\frac{\partial\ub}{\partial t} &=& -\ub\cdot\nabla\ub - \frac{1}{\rho}\nabla p + \gb + \frac{1}{\rho}\Sb_{{\rm ext},\rho\ub}, \\
\frac{\partial p}{\partial t} &=& -\ub\cdot\nabla p - \rho c^2\nabla\cdot\ub -\frac{\partial p}{\partial e}\sum_k q_k\dot\omega_k, + \frac{1}{\rho}\frac{\partial p}{\partial e}S_{{\rm ext},\rho E} + \frac{\partial p}{\partial\rho}S_{{\rm ext},\rho}, \\
\frac{\partial(\rho e)}{\partial t} &=& - \ub\cdot\nabla(\rho e) - (\rho e+p)\nabla\cdot\ub - \sum_k \rho q_k\dot\omega_k + \nabla\cdot\kappa\nabla T + S_{{\rm ext},\rho E}, \\
\frac{\partial A_k}{\partial t} &=& -\ub\cdot\nabla A_k + \frac{1}{\rho}S_{{\rm ext},\rho A_k}, \\
\frac{\partial X_k}{\partial t} &=& -\ub\cdot\nabla X_k + \dot\omega_k + \frac{1}{\rho}S_{{\rm ext},\rho X_k}, \\
\frac{\partial Y_k}{\partial t} &=& -\ub\cdot\nabla Y_k + \frac{1}{\rho}S_{{\rm ext},\rho Y_k}.
\end{eqnarray}

In the code we also carry around $T$ in the primitive state vector.
All of the primitive variables are derived from the conservative state
vector, as described in Section \ref{Sec:Compute Primitive Variables}.
The ordering of the elements within $\Qb$ is defined (in 3D) by
\begin{itemize}
\item QRHO: $\rho$
\item QU: $u$
\item QV: $v$
\item QW: $w$
\item QPRES: $p$
\item QREINT: $\rho e$
\item QTEMP: $T$
\item QFA: $A_1$, the first advected quantity
\item QFS: $X_1$, the first species
\item QFX: $Y_1$, the first auxiliary variable
\end{itemize}
The full primitive variable form (without the advected or auxiliary quantities) is
\begin{equation}
\frac{\partial\Qb}{\partial t} + \sum_d \Ab_d\frac{\partial\Qb}{\partial x_d} = \Sb_{\Qb}.
\end{equation}
For example, in 2D:
\begin{equation}
\left(\begin{array}{c}
\rho \\
u \\
v \\
p \\
\rho e \\
X_k
\end{array}\right)_t
+
\left(\begin{array}{cccccc}
u & \rho & 0 & 0 & 0 & 0 \\
0 & u & 0 & \frac{1}{\rho} & 0 & 0 \\
0 & 0 & u & 0 & 0 & 0 \\
0 & \rho c^2 & 0 & u & 0 & 0 \\
0 & \rho e + p & 0 & 0 & u & 0 \\
0 & 0 & 0 & 0 & 0 & u
\end{array}\right)
\left(\begin{array}{c}
\rho \\
u \\
v \\
p \\
\rho e \\
X_k
\end{array}\right)_x
+
\left(\begin{array}{cccccc}
v & 0 & \rho & 0 & 0 & 0 \\
0 & v & 0 & 0 & 0 & 0 \\
0 & 0 & v & \frac{1}{\rho} & 0 & 0 \\
0 & 0 & \rho c^2 & v & 0 & 0 \\
0 & 0 & \rho e + p & 0 & v & 0 \\
0 & 0 & 0 & 0 & 0 & v
\end{array}\right)
\left(\begin{array}{c}
\rho \\
u \\
v \\
p \\
\rho e \\
X_k
\end{array}\right)_y
=
\Sb_\Qb
\end{equation}
The eigenvalues are:
\begin{equation}
{\bf \Lambda}(\Ab_x) = \{u-c,u,u,u,u,u+c\}, \qquad {\bf \Lambda}(\Ab_y) = \{v-c,v,v,v,v,v+c\} .
\end{equation}
The right column eigenvectors are:
\begin{equation}
\Rb(\Ab_x) =
\left(\begin{array}{cccccc}
1 & 1 & 0 & 0 & 0 & 1 \\
-\frac{c}{\rho} & 0 & 0 & 0 & 0 & \frac{c}{\rho} \\
0 & 0 & 1 & 0 & 0 & 0 \\
c^2 & 0 & 0 & 0 & 0 & c^2 \\
h & 0 & 0 & 1 & 0 & h \\
0 & 0 & 0 & 0 & 1 & 0 \\
\end{array}\right),
\qquad
\Rb(\Ab_y) =
\left(\begin{array}{cccccc}
1 & 1 & 0 & 0 & 0 & 1 \\
0 & 0 & 1 & 0 & 0 & 0 \\
-\frac{c}{\rho} & 0 & 0 & 0 & 0 & \frac{c}{\rho} \\
c^2 & 0 & 0 & 0 & 0 & c^2 \\
h & 0 & 0 & 1 & 0 & h \\
0 & 0 & 0 & 0 & 1 & 0 \\
\end{array}\right).
\end{equation}
The left row eigenvectors, normalized so that $\Rb_d\cdot\Lb_d = \Ib$ are:
\begin{equation}
\Lb_x =
\left(\begin{array}{cccccc}
0 & -\frac{\rho}{2c} & 0 & \frac{1}{2c^2} & 0 & 0 \\
1 & 0 & 0 & -\frac{1}{c^2} & 0 & 0 \\
0 & 0 & 1 & 0 & 0 & 0 \\
0 & 0 & 0 & -\frac{h}{c^2} & 1 & 0 \\
0 & 0 & 0 & 0 & 0 & 1 \\
0 & \frac{\rho}{2c} & 0 & \frac{1}{2c^2} & 0 & 0
\end{array}\right),
\qquad
\Lb_y =
\left(\begin{array}{cccccc}
0 & 0 & -\frac{\rho}{2c} & \frac{1}{2c^2} & 0 & 0 \\
1 & 0 & 0 & -\frac{1}{c^2} & 0 & 0 \\
0 & 1 & 0 & 0 & 0 & 0 \\
0 & 0 & 0 & -\frac{h}{c^2} & 1 & 0 \\
0 & 0 & 0 & 0 & 0 & 1 \\
0 & 0 & \frac{\rho}{2c} & \frac{1}{2c^2} & 0 & 0
\end{array}\right).
\end{equation}
