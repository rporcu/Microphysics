In \mfix\ the gas-phase governing equations for mass and momentum
conservation are similar to those in traditional gas-phase CFD
but with additional coupling terms due to drag from the solids–phase.
The solids-phase is modeled using discrete particles via the
\emph{Discrete Element Method} (DEM). In what follows, we provide
a briefly summary of the governing equations solved in \mfix\ .
The interested user is directed to the 
\href{https://mfix.netl.doe.gov/download/mfix/mfix_current_documentation/dem_doc_2012-1.pdf}{MFIX-DEM user's manual}
for a detailed explanations of the models and implementations used in
\mfix\ .

\section{Gas-phase governing equations}
The governing equations, implemented in \mfix\ for the gas-phase mass
and momentum conservation in the absence of phase change, chemical
reactions, growth, aggregation, breakage phenomena, are
\begin{equation}
  \dfrac{\partial (\varepsilon_g\rho_g) }{\partial t} +\nabla\cdot(\varepsilon_g\rho_g\boldsymbol v_g) = 0,
\end{equation}
and 
\begin{equation}
  \dfrac{D}{D t}(\varepsilon_g\rho_g\boldsymbol v_g) =
  \nabla\cdot \overline{\overline{S}}_g + \varepsilon_g\rho_g\boldsymbol g - \sum\limits_{m=1}^M \boldsymbol I _{gm} = 0.
\end{equation}
In the above equations, $\varepsilon_g$ is the gas-phase volume
fraction, $\rho_g$ is the thermodynamic density of
the gas phase, $\boldsymbol v_g$ is the volume–averaged gas-phase
velocity and  $\boldsymbol I_{gm}$ is the momentum transfer term
between the gas and the m-th solid phase. The gas-phase stress tensor 
$\overline{\overline{S}}_g$ is given by
\begin{equation}
  \overline{\overline{S}}_g = -P_g  \overline{\overline{I}} +  \overline{\overline{\tau}}_g
\end{equation}
where $P_g$ is the gas-phase pressure and $\overline{\overline{\tau}}_g$ the gas-phase shear
stress tensor defined as
\begin{equation}
  \overline{\overline{\tau}}_g = 2\mu_g \overline{\overline{D}}_g  +\lambda_g \nabla\cdot Tr(\overline{\overline{D}}_g )
  \overline{\overline{I}}.
\end{equation}
In the gas-stress tensor definition, $\mu_g$ and $\lambda_g$ are the dynamic
and second coefficients of viscosity respectively, and
$\overline{\overline{D}}_g$ the strain-rate tensor:
\begin{equation}
  \overline{\overline{D}}_g = \dfrac{1}{2}\left[\nabla\boldsymbol v_g + (\nabla\boldsymbol v_g)^T\right].
\end{equation}
